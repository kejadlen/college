%%% George J. Schaeffer
%%% Homework Template (v.05)

%% Begin Header %%

%% Document Settings

\documentclass[11pt]{article}
\usepackage[left=1in,top=1in,right=1in,nohead,nofoot]{geometry}
%\oddsidemargin 0cm
%\topmargin -2cm
%\textwidth 16.5cm
%\textheight 23.5cm

%% End Header %%

\begin{document}

%% Problem Environments, Subject to Customization %%

\newcounter{problem}
\newcounter{superproblem}
\refstepcounter{superproblem}
\renewcommand{\thesuperproblem}{\Roman{superproblem}}

\newenvironment{problem}[1]
{\noindent{\large \textbf{Problem #1}}\\\noindent\begin{itshape}}
{\end{itshape}\medskip}

\newenvironment{problemunnum}
{\refstepcounter{problem}\noindent{\large \textbf{Problem \thesuperproblem}}\\\noindent\begin{itshape}}
{\end{itshape}\medskip}

\newenvironment{problemcit}[1]
{\refstepcounter{problem}\noindent{\large \textbf{Problem \theproblem}: #1}\\\noindent\begin{itshape}}
{\end{itshape}\medskip}

\newenvironment{subproblem}[1]
{\noindent\textbf{Part (#1)}: \begin{itshape}}
{\end{itshape}\medskip}

\newenvironment{ps}[1]
{\bigskip}
{\bigskip}

\newenvironment{soln}
{\noindent}
{\medskip}

\newcommand{\prbsans}{\refstepcounter{problem}\noindent{\large\textbf{Problem \thesuperproblem-\theproblem}}\medskip}
\newcommand{\secheading}[1]{\noindent\textsc{\large #1}}

%% Standard Heading

\begin{flushright}
{Alpha Chen}\\{CS4204 Computer Graphics}\\{Professor Ehrich}\\{Assignment 2}\\{September 10th, 2005}\end{flushright}

%% Document! %%

\begin{problemunnum}
Consider in this homework two directed line segments $L_1$ and $L_2$ as follows:

\[L_1: (10,20) \textrm{to} (-40,10)\]
\[L_2: (-10,5) \textrm{to} (10,10)\]

\begin{problem}
\begin{problemcit}{}Express infinite lines $L_1$ and $L_2$ in slope-intercept form.\end{problemcit}


\end{problem}

\end{problemunnum}

\end{document}
