%% Begin Header %%

%% Document Settings

\documentclass[11pt]{article}
\oddsidemargin 0cm
\topmargin -2cm
\textwidth 16.5cm
\textheight 23.5cm

%% End Header %%

\begin{document}

%% Problem Environments, Subject to Customization %%

\newcounter{problem}
\newcounter{superproblem}
\refstepcounter{superproblem}
\renewcommand{\thesuperproblem}{\Roman{superproblem}}

\newenvironment{problem}[1]
{\noindent{\large \textbf{Problem #1}}\\\noindent\begin{itshape}}
{\end{itshape}\medskip}

\newenvironment{problemunnum}
{\refstepcounter{problem}\noindent{\large \textbf{Problem \thesuperproblem}}\\\noindent\begin{itshape}}
{\end{itshape}\medskip}

\newenvironment{problemcit}[1]
{\refstepcounter{problem}\noindent{\large \textbf{Problem \theproblem}: #1}\\\noindent\begin{itshape}}
{\end{itshape}\medskip}

\newenvironment{subproblem}[1]
{\noindent\textbf{Part (#1)}: \begin{itshape}}
{\end{itshape}\medskip}

\newenvironment{ps}
{\bigskip}
{\bigskip}

\newenvironment{soln}
{\noindent}
{\medskip}

\newcommand{\prbsans}{\refstepcounter{problem}\noindent{\large\textbf{Problem \thesuperproblem-\theproblem}}\medskip}
\newcommand{\secheading}[1]{\noindent\textsc{\large #1}}

%% Standard Heading

\begin{flushright}
{Alpha Chen}\\{CS 3304 Comparative Languages}\\{Professor Edwards}\\{HW 04}\\{September 21st, 2005}\end{flushright}

%% Document! %%

\begin{ps}
\begin{problemcit}{}Write a Scheme function that takes a list as a parameter and returns a copy of the list with the last element removed.\end{problemcit}
\begin{soln}
\begin{verbatim}
(define (butlast lst)
  (reverse (rest (reverse lst))))
\end{verbatim}
\end{soln}
\end{ps}

\begin{ps}
\begin{problemcit}{}Write your own implementation of the Scheme built-in function \verb|fold-right|.\end{problemcit}
\begin{soln}
\begin{verbatim}
(define (fold-right binary-procedure initial-value list-arg)
  (if (= 1 (length list-arg))
    (binary-procedure (car list-arg) initial-value)
    (binary-procedure
      (car list-arg)
      (fold-right
        binary-procedure
        initial-value
        (cdr list-arg)))))
\end{verbatim}
\end{soln}
\end{ps}

\begin{ps}
\begin{problemcit}{Chapter 5, Problem 8}
\begin{subproblem}{a}Assuming static scoping, which declaration of \verb|x| is the correct one for a reference to \verb|x| in the following:
\renewcommand{\labelenumi}{\roman{enumi}}
\begin{enumerate}
\item \verb|Sub1|
\item \verb|Sub2|
\item \verb|Sub3|
\end{enumerate}
\end{subproblem}
\begin{subproblem}{b}Repeat part a, but assume dynamic scoping.\end{subproblem}
\end{problemcit}
\begin{soln}
\renewcommand{\labelenumi}{\alph{enumi}}
\renewcommand{\labelenumii}{\roman{enumii}}
\begin{enumerate}
\item \begin{enumerate}
  \item \verb|Sub1|
  \item \verb|Sub1|
  \item \verb|Main|
  \end{enumerate}
\item \begin{enumerate}
  \item \verb|Sub1|
  \item \verb|Sub1|
  \item \verb|Sub1|
  \end{enumerate}
\end{enumerate}
\end{soln}
\end{ps}

\begin{ps}
\begin{problemcit}{Chapter 5, Problem 9}What value of \verb|x| is printed in procedure \verb|Sub1| under static and dynamic scoping rules?\end{problemcit}

\begin{soln}
Under static scoping, 5 is printed. Under dynamic scoping, 10 is printed.
\end{soln}
\end{ps}

\end{document}
