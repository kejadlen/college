%% Begin Header %%

%% Document Settings

\documentclass[11pt]{article}
\usepackage{bnf}
\oddsidemargin 0cm
\topmargin -2cm
\textwidth 16.5cm
\textheight 23.5cm

%% End Header %%

\begin{document}

%% Problem Environments, Subject to Customization %%

\newcounter{problem}
\newcounter{superproblem}
\refstepcounter{superproblem}
\renewcommand{\thesuperproblem}{\Roman{superproblem}}

\newenvironment{problem}[1]
{\noindent{\large \textbf{Problem #1}}\\\noindent\begin{itshape}}
{\end{itshape}\medskip}

\newenvironment{problemunnum}
{\refstepcounter{problem}\noindent{\large \textbf{Problem \thesuperproblem}}\\\noindent\begin{itshape}}
{\end{itshape}\medskip}

\newenvironment{problemcit}[1]
{\refstepcounter{problem}\noindent{\large \textbf{Problem \theproblem}: #1}\\\noindent\begin{itshape}}
{\end{itshape}\medskip}

\newenvironment{subproblem}[1]
{\noindent\textbf{Part (#1)}: \begin{itshape}}
{\end{itshape}\medskip}

\newenvironment{ps}
{\bigskip}
{\bigskip}

\newenvironment{soln}
{\noindent}
{\medskip}

\newcommand{\prbsans}{\refstepcounter{problem}\noindent{\large\textbf{Problem \thesuperproblem-\theproblem}}\medskip}
\newcommand{\secheading}[1]{\noindent\textsc{\large #1}}

%% Standard Heading

\begin{flushright}
{Alpha Chen}\\{CS 3304 Comparative Languages}\\{Professor Edwards}\\{HW 02}\\{September 7th, 2005}\end{flushright}

%% Document! %%

\begin{ps}
\begin{problemcit}{Chapter 4, Problem 3} Show a trace of the recursive descent parser given in Section 4.4.1 for the string \verb|a + b * c|\end{problemcit}
\begin{soln}
\begin{verbatim}
Call lex      /* returns a */
Enter <expr>
Enter <term>
Enter <factor>
Call lex      /* returns + */
Exit <factor>
Exit <term>
Call lex      /* returns b */
Enter <term>
Enter <factor>
Call lex      /* returns * */
Exit <factor>
Enter <factor>
Call lex      /* returns c */
Exit <factor>
Enter <factor>
Call lex      /* returns end-of-input */
Exit <factor>
Exit <term>
Exit <expr>
\end{verbatim}
\end{soln}
\end{ps}

\begin{ps}
\begin{problemcit}{Chapter 3, Problem 16} Convert the BNF of Example 3.3 to EBNF.\end{problemcit}
\begin{soln}
  \begin{grammar}
  [(colon){$\rightarrow\ $}]
  [(semicolon){\hspace{-7pt}$|$}]
  [(comma){}]
  [(period){{\\[1ex]\mbox{}\hspace{-2em}}}]
  [(quote){\begin{bf}$\;$}{\end{bf}}]
  [(nonterminal){$\langle$}{$\rangle$}]

  \vspace{-4ex}\mbox{}.
  <assign>: <id> = <expr>.
  <id>: A ;B ;C.
  <expr>: <expr> \{ ( + ; * ) <expr> \} ; ( <expr> ) ; <id>.
  \end{grammar}
\end{soln}
\end{ps}

\begin{ps}
\begin{problemcit}{Chapter 3, Problem 18(e)} Using the virtual machine instructions given in Section 3.5.1.1, give an operational semantic definition of the C \verb|for| loop\end{problemcit}
\begin{soln}
\begin{verbatim}
      i = un_op var
loop: ...
      i = i bin_op var
      if i relop var goto out
      goto loop
out:  ...
\end{verbatim}
\end{soln}
\end{ps}

\begin{ps}
\begin{problemcit}{Chapter 3, Problem 20(b)} Compute the weakest precondition for each of the following sequences of assignment statements and their postconditions:
\begin{verbatim}
  a = 3 * (2 * b + a)
  b = 2 * a - 1
  {b > 5}
\end{verbatim}
\end{problemcit}
\begin{soln}
\begin{verbatim}
2 * a - 1 > 5
2 * a > 6
a > 3

-3 * b = a
-3 * b > 3
{b > -1}
\end{verbatim}
\end{soln}
\end{ps}

\end{document}
