%% Begin Header %%

%% Document Settings

\documentclass[11pt]{article}
\usepackage{graphics}
%\usepackage{amssymb,amsmath,mathdots}
%\usepackage{eucal}
\usepackage{bnf}
\oddsidemargin 0cm
\topmargin -2cm
\textwidth 16.5cm
\textheight 23.5cm

%% End Header %%

\begin{document}

%% Problem Environments, Subject to Customization %%

\newcounter{problem}
\newcounter{superproblem}
\refstepcounter{superproblem}
\renewcommand{\thesuperproblem}{\Roman{superproblem}}

\newenvironment{problem}[1]
{\noindent{\large \textbf{Problem #1}}\\\noindent\begin{itshape}}
{\end{itshape}\medskip}

\newenvironment{problemunnum}
{\refstepcounter{problem}\noindent{\large \textbf{Problem \thesuperproblem}}\\\noindent\begin{itshape}}
{\end{itshape}\medskip}

\newenvironment{problemcit}[1]
{\refstepcounter{problem}\noindent{\large \textbf{Problem \theproblem}: #1}\\\noindent\begin{itshape}}
{\end{itshape}\medskip}

\newenvironment{subproblem}[1]
{\noindent\textbf{Part (#1)}: \begin{itshape}}
{\end{itshape}\medskip}

\newenvironment{ps}
{\bigskip}
{\bigskip}

\newenvironment{soln}
{\noindent}
{\medskip}

\newcommand{\prbsans}{\refstepcounter{problem}\noindent{\large\textbf{Problem \thesuperproblem-\theproblem}}\medskip}
\newcommand{\secheading}[1]{\noindent\textsc{\large #1}}

%% Standard Heading

\begin{flushright}
{Alpha Chen}\\{CS 3304 Comparative Languages}\\{Professor Edwards}\\{HW 01}\\{August 30th, 2005}\end{flushright}

%% Document! %%

\begin{ps}
\begin{problemcit}{(Based on Chapter 2)} What innovation of data structuring was introduced in ALGOL 68 but is often credited to Pascal?\end{problemcit}
\begin{soln}ALGOL 68 provided user-defined data types, although this is often ascribed to Pascal.\end{soln}
\end{ps}

\newpage

\begin{ps}
\begin{problemcit}{Chapter 3, Problem 6} Using the grammar in Example 3.2, show a parse tree and a leftmost derivation for each of the following statements:
  \begin{itemize}
    \item[a.] \verb|A = A * (B + (C * A))|
    \item[b.] \verb|B = C * (A * C + B)|
    \item[c.] \verb|A = A * (B + (C))|
  \end{itemize}
\end{problemcit}
\begin{soln}
  \begin{itemize}
    \item[a.] 
      \begin{verbatim}
<assign> => <id> = <expr>
         => A = <expr>
         => A = <id> * <expr>
         => A = A * <expr>
         => A = A * (<expr>)
         => A = A * (<id> + <expr>)
         => A = A * (B + <expr>)
         => A = A * (B + (<expr>))
         => A = A * (B + (<id> * <expr>))
         => A = A * (B + (C * <expr>))
         => A = A * (B + (C * <id>))
         => A = A * (B + (C * A))
      \end{verbatim}
      \includegraphics{hw01_2_a.png}

\newpage

    \item[b.] 
       \begin{verbatim}
<assign> => <id> = <expr>
         => B = <expr>
         => B = <id> * <expr>
         => B = C * <expr>
         => B = C * (<expr>)
         => B = C * (<id> * <expr>)
         => B = C * (A * <expr>)
         => B = C * (A * <id> + <expr>)
         => B = C * (A * C + <expr>)
         => B = C * (A * C + <id>)
         => B = C * (A * C + B)
      \end{verbatim}
      \includegraphics{hw01_2_b.png}

\newpage
      
    \item[c.]
        \begin{verbatim}
<assign> => <id> = <expr>
         => A = <expr>
         => A = <id> * <expr>
         => A = A * <expr>
         => A = A * (<expr>)
         => A = A * (<id> + <expr>)
         => A = A * (B + <expr>)
         => A = A * (B + (<expr>))
         => A = A * (B + (<id>))
         => A = A * (B + (C))
         => A = A * (B + (C))
         => A = A * (B + (C))
      \end{verbatim}
      \includegraphics{hw01_2_c.png}
  \end{itemize}
\end{soln}
\end{ps}

\begin{ps}
\begin{problemcit}{Chapter 3, Problem 4} Rewrite the BNF of Example 3.4 to add the \verb|++| and \verb|--| unary operators of Java.\end{problemcit}\\
\begin{soln}
  \begin{grammar}
  [(colon){$\rightarrow\ $}]
  [(semicolon){\hspace{-7pt}$|$}]
  [(comma){}]
  [(period){{\\[1ex]\mbox{}\hspace{-2em}}}]
  [(quote){\begin{bf}$\;$}{\end{bf}}]
  [(nonterminal){$\langle$}{$\rangle$}]

  \vspace{-4ex}\mbox{}.
  <assign>: <id> = <new>.
  <id>: A ;B ;C.
  <expr>: <expr> + <term> ;<term>.
  <term>: <term> * <inc\_dec> ;<inc\_dec>.
  <factor>: ( <expr> ) ;<inc\_dec>.
  <inc\_dec>: (++ ;"\verb|--|")<id> ; <id>(++ ;"\verb|--|").
  \end{grammar}
\end{soln}
\end{ps}

\begin{ps}
\begin{problemcit}{}Prove that the following language is ambiguous:
\begin{verbatim}
<sequence> --> <base>
            |  <sequence> <sequence>

            <base>     --> A | C | T | G
\end{verbatim}
\end{problemcit}
\begin{soln}
  The sequence ACT can be produced by two different parse trees, so this language is ambiguous.

  \includegraphics{hw01_4_a.png}\includegraphics{hw01_4_b.png}

\end{soln}
\end{ps}

\end{document}
