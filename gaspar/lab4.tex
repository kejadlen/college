\documentclass[12pt]{article}

\begin{document}
\section{titlepage}
\newpage

\section{Objectives and Procedure}
This experiment involves simple harmonic motion.  A torsion spring will be
used and its motion demonstrated to be SHM.  Further examination will be
used to show that the period is the same, regardless of angular displacement.
Finally, the moment of inertia of the pendulum will be found by modifiying
its moment of inertia.

\newpage

\section{Motion of a Torsion Pendulum}
\begin{center}
(The two printouts follow this section.)
\end{center}

\paragraph{a)}
See printout.

\paragraph{b)}
At time 0.250 s, the angular position, velocity, and acceleration is reported
as -1.343 rad, -1.842 rad/s, and 13.498 rad/s/s, respectively.  The equation 
obtained for the motion of the pendulum is \(\theta=A\cos(\omega t + \phi)\),
where \(A = 1.45\) rad, \(\omega = 3.142\) rad/s, and \(\phi = 2\) rad.  
Plugging in these values for \(t = 0.25\) s:

\[\theta=1.45 \cos(3.142 t + 2) = -1.359\]

The velocity and acceleration functions can be found by simple manipulation
of the position function.  Respectively, \(d\theta/dt =
-A\omega\sin(\omega t + \phi)\) and \(d^2\theta/{dt}^2 = -A\omega^2\cos(
\omega t + \phi)\).  Again plugging in for t:

\[d\theta/dt = -1.45\cdot 3.142\sin(3.142t + 2) = -1.579\]
\[d^2\theta/{dt}^2 = -1.45\cdot 3.142^2\cos(3.142t + 2) = 13.263\]

\newpage

\section{Amplitude Dependence of Frequency}

\paragraph{a)}
\begin{tabular}{|c|c|}
\hline
amplitude & measured angular frequency\\
\hline
$90^\circ$ & 0.511 Hz \\
$45^\circ$ & 0.511 Hz \\
$20^\circ$ & 0.511 Hz \\
\hline
\end{tabular}

It can be seen from the data collected that regardless of the amplitude,
the angular frequency is a constant value.  In this case, it is 0.511 Hz.

\newpage

\section{Relationship of Frequency to Mechanical Parameters}

\paragraph{A-a)}

The period of oscillation of just the pendulum was 1.958 seconds, or 0.511
Hz.  A disc ring was placed on top of the platform in order to add to the
moment of inertia of just the pendulum.  The ring had an inner radius of 
0.0317 m, and an outer radius of 0.0762 m.  Its mass was measured to be 
0.5160 kg.  The moment of inertia of such a disc (with uniform density) is
equal to \(\frac{1}{2}M({R_i}^2 + {R_o}^2)\).  Thus, the moment of inertia
of the disc was found to be \(\frac{1}{2}0.0762(0.0317^2 + 0.0762^2)
= 0.0018\) kg$\cdot$m$^2$.  Earlier on, a formula was derived such that the moment
of inertia of a platform in SHM could be found by adding a known moment of
inertia and comparing the initial period against the new period:

\[I_0 = \frac{I_c}{1+(T_1/T_0)^2}\]

$T_1$ was found to be 2.804 seconds.  With this data,

\[I_0 = \frac{0.0018}{1+(2.804/1.958)^2} = 0.0017 \textrm{kg$\cdot$m$^2$}\]

\paragraph{B-a)}
\begin{center}
(The plot follows this section.)
\end{center}

Since \(\theta = - \frac{Rmg}{\kappa}\), \(\kappa = - \frac{Rmg}{\theta}\).
The projected line seems to fall through the point where mass \(= 0.01\) kg
and angle \(= 0.25\) rad.  Plugging into the above equation gives us
\(\kappa = - \frac{0.05715\cdot -9.8\cdot 0.01}{0.25} = 0.0224\) N m/rad.

\paragraph{C-a)}
From the above section, the torsion constant of the wire was calculated to
be 0.0224 N m/rad.  The moment of inertia was calculated to be 0.0017 kg
m$^2$, also found above.  With \(\omega = \sqrt{\frac{\kappa}{I_0}}\), the
angular velocity can be calculated to be \(\sqrt{\frac{0.0224}{0.0017}} =
3.63\) rad/s.  However, the angular velocity was observed to be actually
3.21 rad/s in part two.  This is a percent difference of \(100\cdot
|\frac{C - M}{M}| = 11.5\) percent.  

\newpage

\section{Conclusions}

\paragraph{}

This experiment shows several interesting facts about simple harmonic motion
with a torsion pendulum.  These results may be extrapolated to simple 
harmonic motion in general.  It is first seen that the change in position
is indeed \(\theta = A\cos(\omega t + \phi)\).   By taking the derivative of
this equation, the velocity function can be found.  And by taking the derivative
of that, the acceleration function can also be found.  The frequency
is then observed to be independent of the position.  Finally, it is possible
to find the moment of inertia of the initial pendulum from an added moment
of inertia and the ratio of the two periods.

\end{document}