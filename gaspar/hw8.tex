%%% Alpha Chen

\documentclass[11pt]{article}
\usepackage[pdftex]{graphics}
\usepackage{graphics}
\usepackage{fancyhdr}
\usepackage{amssymb,amsmath}
\oddsidemargin0cm \topmargin-2cm \textwidth 16.5cm \textheight
23.5cm

\begin{document}
\input{epsf.sty}

\newcounter{problem}

\newenvironment{ps}
{\noindent}
{\bigskip\bigskip}

\newenvironment{problem}
{\refstepcounter{problem}\noindent{\large \textbf{Problem \theproblem}}\\\noindent\begin{itshape}}
{\end{itshape}\medskip}

\newenvironment{problemcit}[1]
{\refstepcounter{problem}\noindent{\large \textbf{Problem \theproblem}: #1}\\\noindent\begin{itshape}}
{\end{itshape}\medskip}
\newenvironment{subproblem}[1]
{\noindent\textbf{Part (#1)}\\\noindent\begin{itshape}}
{\end{itshape}\medskip}

\medskip \medskip
\begin{flushright}
Alpha Chen \\
Homework 8 \\
December 7th, 2004
\end{flushright}

\sffamily{

\begin{ps}\begin{problemcit}{21.3.2(a)-(b)(i)}\end{problemcit}

\begin{subproblem}{a}See attached.\end{subproblem}

\begin{subproblem}{b}See attached.\end{subproblem}

\end{ps}

\begin{ps}\begin{problemcit}{21.6.3}\end{problemcit}

The system in figure P21.6.3 is a cascade interconnection structure. $H_1(z)$ is a transfer function in Direct Form I, and $H_2(z)$ is a transfer function in Direct Form II. Both of these transfer functions can be found:

\begin{eqnarray*}
H_1(z) & = & \frac{1}{-0.81z^{-2}} \\
H_2(z) & = & 1+2.41z^{-1}+2.41z^{-2}+z^{-3}
\end{eqnarray*}

Because $H(z) = H_1(z)H_2(z)$ for cascade interconnections,

\begin{equation*}
H(z) = \frac{1+2.41z^{-1}+2.41z^{-2}+z^{-3}}{-0.81z^{-2}}\text{.}
\end{equation*}

Therefore, the difference equations for this system are

\begin{eqnarray*}
w(n) & = & x(n) - 0.81w(n-2) \text{ and} \\
y(n) & = & w(n) + 2.41w(n-1) + 2.41w(n-2) + w(n-3) \text{.}
\end{eqnarray*}

\end{ps}

\begin{ps}\begin{problemcit}{23.1.4(i)}\end{problemcit}

\begin{eqnarray*}
x(n) & = & \sin(2n) \\
& = & \cos(2n - \frac{\pi}{2}) \\
& = & \cos(\Omega_in) \\
\Omega_i & = & 2
\end{eqnarray*}

\begin{eqnarray*}
y_{ss}(n) & = & \big|H(\Omega_i)\big|\cos\left(\Omega_in+\angle H(\Omega_in)\right) \\
& = & 0.5\cos\left(2n-\frac{\pi}{2}+2\right)
\end{eqnarray*}

\end{ps}

\begin{ps}\begin{problemcit}{23.2.3}\end{problemcit}

The notch in Figure P23.2.3 has cutoff frequencies of $\Omega=0.9$ and $\Omega=1.3$.

\begin{subproblem}{a}For $f_s=1$ kHz, $\Omega = f_s\Omega$. Thus, the continuous cutoff frequencies are 900 and 1300 Hz.\end{subproblem}

\begin{subproblem}{b}For $f_s=4$ kHz, $\Omega = f_s\Omega$. Thus, the continuous cutoff frequencies are 2800 and 5200 Hz.\end{subproblem}

\begin{subproblem}{c}The input signal for this filter is
\begin{equation*}
x(n) = 1.5\cos(400\pi t) + 0.3\cos(800\pi t + 1.2)
\end{equation*}

For $f_s=1$ kHz, $\Omega_i$ is $400\pi/1000 = 0.4\pi = 1.26$ for the first term and $800\pi/1000 = 0.8\pi = 2.52$ for the second term. The amplitude of the steady state response can be found to be 0.9 for the first term and 1 for the second term.

For $f_s=4$ kHz, $\Omega_i$ is $400\pi/4000 = 0.1\pi = 0.31$ for the first term and $800\pi/4000 = 0.2\pi = 0.62$ for the second term. The amplitude of the steady state response can be found to be 1 for both the first term and the second term.

\end{subproblem}

\end{ps}

\end{document}
